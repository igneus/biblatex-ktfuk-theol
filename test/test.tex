\documentclass[12pt, a4paper, landscape]{article}
\usepackage[margin=1.5cm]{geometry}

\usepackage{fontspec}
\setmainfont[Ligatures={TeX}]{TeXGyreTermes}

\usepackage[czech]{babel}
\usepackage{csquotes}

\usepackage[backend=biber, style=ktfuk-theol]{biblatex}

%\usepackage[hidelinks]{hyperref}

% nahrat bibliografii:
\bibliography{testbibliografie}

\title{Test stylu bibliografie a citací ktfuk-theol}

\newcommand{\vzduch}{\vspace{8mm}}

\begin{document}

\maketitle

\textsf{ktfuk-theol} je styl bibliografie a citací pro BibLaTeX,
vyvíjený s cílem dosáhnout plného souladu s interními předpisy
Katolické teologické fakulty University Karlovy v Praze 
o bibliografických údajích a citacích ve studentských pracích
v oboru teologie.\footcite[8-12]{ktfukpravidla}

Tento dokument slouží k ověření (ne)správnosti fungování tohoto stylu
srovnávací metodou.

\section{Srovnávací test generování bibliografických záznamů}

Níže je pro každý ze vzorových bibliografických 
záznamů uvedených v normě nejprve \uv{natvrdo} ručně vysázen vzor
a pod ním ten samý záznam vygenerovaný BibLaTeXem z připravené
bibliografické databáze \textsf{testbibliografie.bib}.

\subsubsection{Monografie}

\setlength{\parindent}{0pt}

\indent%
RATZINGER, Joseph. 
\emph{Eschatologie: smrt a věčný život.} 
2. vydání. 
Brno: Barrister \& Principal, 
2004.

\cite{eschatologie}

\vzduch

FIORENZA, Francis S. – GALVIN, John P. 
\emph{Systematická teologie: římskokatolická perspektiva.} 
1. sv. 
Brno: Centrum pro studium demokracie a kultury, 
1996.

\cite{fiorenza}

\subsubsection{Kapitola v monografii}

MADEC, Goulven. 
Aurelius Augustinus: křesťanství jako naplnění platonismu.
In 
KARFÍK, Filip – NĚMEC, Václav – VILÍM, František (ed.). 
\emph{Křesťanství a filosofie: postavy latinské tradice: 
Augustinus, Boëthius, Jan Eriugena, 
Anselm z Canterbury, Tomáš Akvinský, Jan Duns Scotus.} 
Praha: 
Česká křesťanská akademie,
1994, 
s. 9–26.

\cite{krestanstviplatonismus}

\subsubsection{Článek v časopise}

KASPER, Walter. 
Kristovo nanebevstoupení – historie a theologický význam.
\emph{Mezinárodní katolická revue Communio} 
1999, 
roč. 3, č. 3, 
s. 223–232.

\cite{nanebevstoupeni}

\vzduch

BALTHASAR, Hans Urs von. 
První v novém světě.
\emph{Mezinárodní katolická revue Communio} 
1999, 
roč. 3, č. 3, 
s. 297–302.

\cite{prvni}

\vzduch

MANDL, Antonín. 
Pouť na Velehrad L. P. 1968.
\emph{Katolické noviny} 
1968, 
16. 5., 
roč. 20, č. 20, 
s. 1–2.

\cite{pout}

\subsubsection{Elektronický zdroj}

Český statistický úřad. 
\emph{Grafy populačního vývoje 1946–2004} 
[2006-03-18].
<http://www.czso.cz>.

\cite{populacnivyvoj}

\vzduch

SKOBLÍK, Jiří. 
\emph{Setkání empirie se spekulací v etické argumentaci církve} 
(4.9.2002)
[2006-11-16]. 
<http://ktf.cuni.cz/\textasciitilde skoblik/empir\_spekul.htm>.

\cite{empiriespekulace}

\vzduch

\emph{Prohlášení arcibiskupství pražského k papežské návštěvě ČR} 
(16.11.2006) 
[2006-11-16]. 
<http://tisk.cirkev.cz/z-domova/prohlaseni-arcibiskupstvi-prazskeho-k-papezske-navsteve-cr.html>.

\cite{papezskanavsteva}

\end{document}
